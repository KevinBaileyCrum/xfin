\documentclass[a4paper, 11pt]{article}
\usepackage{fullpage} % changes the margin
\setlength\parindent{24pt} % for indenting with \par
\usepackage{sectsty}

\sectionfont{\fontsize{11}{11}\selectfont}

% for inserting image
\usepackage{graphicx}
\graphicspath{}

% listings for inserting code --
\usepackage{color}

\definecolor{dkgreen}{rgb}{0,0.6,0}
\definecolor{gray}{rgb}{0.5,0.5,0.5}
\definecolor{mauve}{rgb}{0.58,0,0.82}


% for bold line under section heading
\usepackage{titlesec}
\titleformat{\section}
  {\normalfont\Large\bfseries}{\thesection}{1em}{}[{\titlerule[0.8pt]}]

%%%%%%%%%%%%%%%%%%%%%%%%%%%%%%%%%% 
\begin{document}
%%%%%%%%%%%%%%%%%%%%%%%%%%%%%%%%%%


\begin{titlepage} % Suppresses headers and footers on the title page

	\centering % Centre everything on the title page
	
	\scshape % Use small caps for all text on the title page
	
	\vspace*{\baselineskip} % White space at the top of the page
	
	%------------------------------------------------
	%	Title
	%------------------------------------------------
	
	\rule{\textwidth}{1.6pt}\vspace*{-\baselineskip}\vspace*{2pt} % Thick horizontal rule
	\rule{\textwidth}{0.4pt} % Thin horizontal rule
	
	\vspace{0.75\baselineskip} % Whitespace above the title
	
	{\LARGE Using Xfinity On Campus:\\} % Title
	\vspace{0.5\baselineskip} % Whitespace below the editor list
  {\scshape How to Use Internet \\ To Access Television}
	
	\vspace{0.75\baselineskip} % Whitespace below the title
	
	\rule{\textwidth}{0.4pt}\vspace*{-\baselineskip}\vspace{3.2pt} % Thin horizontal rule
	\rule{\textwidth}{1.6pt} % Thick horizontal rule
	
	\vspace{2\baselineskip} % Whitespace after the title block
	
	%------------------------------------------------
	%	Subtitle
	%------------------------------------------------
	
	
	\vspace*{3\baselineskip} % Whitespace under the subtitle
	
	%------------------------------------------------
	%	Editor(s)
	%------------------------------------------------
	
	\vspace{0.5\baselineskip} % Whitespace before the editors
	
	\vspace{0.5\baselineskip} % Whitespace below the editor list
	
	{\scshape ResNet \\ 831-459-4638 \\
          \textit{Summer Hours}: 10AM - 12PM, 1PM - 5PM, Monday - Friday \\
          Rachel Carson College (formerly College 8) \\
          \vfill
          \vfill
          \vfill
          \textit{Summer Hours End 8/31}\\ 
          We will reopen and resume welcome week hours on 9/20.  Check our
          website https://resnet.ucsc.edu/ for more information. 
  } % Editor affiliation
	
	\vfill % Whitespace between editor names and publisher logo
	

\end{titlepage}

% ----------------------------------------------------------------------------



\section*{
  Connecting to Xfinity On Campus with a computer
}

You can connect to Xfinity On Campus with a computer or other device that
talks to the internet. With Xfinity On Campus, you may
use a computer and stream it to your TV either via connecting through a 
cable (hdmi, vga + audio etc), through a streaming device such as a Chromecast
,or by simply watching on your computer screen.
From here you may stream on demand, save shows, and 
watch live TV. If you would like to know about your television service 
via a coaxial connection, see the last section of the article.\\
{\newline}
\textbf{How to Connect}

In order to connect to Xfinity On Campus you must be connected to one of the
campus networks.  This includes: UCSC-Guest, Eduroam, ResWiFi, or through a 
wired Ethernet connection.\\
{\newline}
\textbf{Logging into Xfinity On Campus}
\begin{itemize}
  \item To connect to Xfinity On Campus from your computer, ensure that your device  is connected to the internet. 
  \item From your browser, navigate to https://www.xfinityoncampus.com/.  Enter University of California Santa Cruz in the 
  \textbf{Search For Your School} Bar.
  Note: as of now a private browsing or incognito window will not work with
  Xfinity's streaming service.
\end{itemize}
\includegraphics[width=\linewidth, height=\textheight, keepaspectratio]{welcome.png}
\begin{itemize}
  \item Enter your cruz credentials, afterwards you will be directed to a new URL \\
  https://www.xfinity.com/stream/.  Note: you may get taken to an intermediary
  information release page from 
  https://login.ucsc.edu/  You may  accept and proceed.

  \item Once you are on logged in to the stream site, the two fields most
  pertinent to watching television are \textbf{Channels} (peach) and 
  \textbf{Watch Now}
  (red).
\end{itemize}

\includegraphics[width=\linewidth, keepaspectratio]{home.png}
\subsection*{
  Channels 
}

Channels is a useful interface that shows the current channel lineup. 
When a desired channel or show is found, select \textbf{Watch Now}. You
will be directed to the same URL as if you had selected \textbf{Watch Now} on
the Xfinity stream page which will be described below.

\subsection*{
  Watch Now
}

Watch Now allows you stream from available programs, watch live TV, and record television programs.  To reiterate, this page will not work in a private 
browsing window and you must have Flash enabled.  

\begin{itemize}
  \item Streaming is as simple as browsing through available on demand 
  programs and selecting the one you want to watch.
  \item Live TV can be found by selecting the \textbf{Live TV} drop down 
  header menu item and selecting \textbf{All Channels}.  Once you click
  on a channel, you may either watch, record, or view more information
  about the program.
\end{itemize}
\includegraphics[width=\linewidth, keepaspectratio]{seas.png}
\begin{itemize}
  \item Recording shows is possible with the record option.  Once selected
  you may choose to record one episode or the entire series.  Recordings 
  can be viewed/managed by selecting the \textbf{Saved} drop down header
  menu item and selecting \textbf{Recordings}.
\end{itemize}
\newpage
\section*{
  Connecting to the Television service using Coax
}
For users in locations with access to television over coaxial cables, the new service will be accessed in the same way as the old service. Simply connect the coaxial cable from the wall port or splitter into the back of your television. A channel scan may be required to access the lineup. See more specific instructions below.

\subsection*{Coaxial Cable and port:}
\includegraphics[width=\linewidth, keepaspectratio]{coax.png}

\subsection*{Channel Scans}
For users who previously used the coaxial TV service, a channel scan may be required to access the new lineup. This may also be required when connecting a new television to coax for the first time. Instructions on how to run a channel scan vary based on manufacturer and model. The channel scan option usually be reached through the television’s menus. For more specific instructions check with your television’s manual or manufacturer.

\subsection*{Digital converters}
For TVs built before 2007, or if your TV lacks the capability to access digital television signals via coax, you may need a digital converter box. Some manufacturers do not include digital converters even on modern TVs to save costs. If a channel scan only reveals between 1 and 6 channels, it is likely that your television will need a digital converter box. A channel scan will likely need to be run from the converter box in order to pick up the channels. 


\end{document}

